\begin{tabular}{|p{0.450\textwidth}|p{0.450\textwidth}|}
  \hline
  \multicolumn{1}{|c|}{\textbf{\large{Requête}}} & \multicolumn{1}{|c|}{\textbf{\large{Résultat}}} \\
  \hline
  \multicolumn{2}{|c|}{\textbf{\large\makecell{Sélectionner les instances de ServeurWeb}}} \\
  \hline

  \begin{lstlisting}[style=codeStyle]
SELECT ?serveurWeb 
WHERE {
    ?serveurWeb rdf:type <http://example.com/ServeurWeb>.
}
\end{lstlisting}

&

  \begin{lstlisting}[style=resultStyle]
------------------------------
| serveurWeb                 |
==============================
| <http://example.com/Nginx> |
| <http://example.com/IIS>   |
------------------------------
\end{lstlisting} \\
  \hline

  \multicolumn{2}{|c|}{\textbf{\large\makecell{Décrire l'instance "Nginx"}}} \\
  \hline
  \begin{lstlisting}[style=codeStyle]
DESCRIBE <http://example.com/Nginx>
\end{lstlisting}

&
  \begin{lstlisting}[style=resultStyle]
(graph
 (triple <http://example.com/Nginx> rdf:type owl:NamedIndividual)
 (triple <http://example.com/Nginx> rdf:type <http://example.com/ServeurWeb>)
 (triple <http://example.com/Nginx> <http://example.com/version> "1.23.4")
)
\end{lstlisting} \\
  \hline

  \multicolumn{2}{|c|}{\textbf{\large\makecell{Demander s'il existe une instance de Client}}} \\
  \hline
  \begin{lstlisting}[style=codeStyle]
ASK {
    ?nav rdf:type <http://example.com/NavigateurWeb>.
}
\end{lstlisting}

&

  \begin{lstlisting}[style=resultStyle]
yes
\end{lstlisting} \\
\hline

\end{tabular}

\newpage

\begin{tabular}{|p{0.450\textwidth}|p{0.450\textwidth}|}

  \hline
  \multicolumn{2}{|c|}{\textbf{\large\makecell{Construire un graphe des ServeurWeb et leur version}}} \\
  \hline
  \begin{lstlisting}[style=codeStyle]
CONSTRUCT {
    ?serveurWeb <http://example.com/version> ?version.
}
WHERE {
    ?serveurWeb <http://example.com/version> ?version .
    ?serveurWeb rdf:type <http://example.com/ServeurWeb>.
}
\end{lstlisting}

&

  \begin{lstlisting}[style=resultStyle]
(graph
 (triple <http://example.com/Nginx> <http://example.com/version> "1.23.4")
 (triple <http://example.com/IIS> <http://example.com/version> "2.34.5")
)
\end{lstlisting}
  \\
  \hline

  \multicolumn{2}{|c|}{\textbf{\large\makecell{Insérer une nouvelle instance de ServeurWeb}}} \\
  \hline
  \begin{lstlisting}[style=codeStyle]
INSERT DATA {
    <http://example.com/Apache> rdf:type <http://example.com/ServeurWeb>.
    <http://example.com/Apache> <http://example.com/version> "3.45.6".
}
\end{lstlisting}

&

  \begin{lstlisting}[style=resultStyle]
  SPARQL update executed correctly
  \end{lstlisting}

  \\
  \hline
  \multicolumn{2}{|c|}{\textbf{\large\makecell{Supprimer une instance de ServeurWeb}}} \\
  \hline
  \begin{lstlisting}[style=codeStyle]
DELETE {
    <http://example.com/Apache> rdf:type <http://example.com/ServeurWeb>.
}
WHERE {
    <http://example.com/Nginx> rdf:type <http://example.com/ServeurWeb>.
}
\end{lstlisting}

&

\begin{lstlisting}[style=resultStyle]
  SPARQL update executed correctly
  \end{lstlisting}
  \\
  \hline
\end{tabular}